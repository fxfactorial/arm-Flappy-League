\documentclass[11pt]{article}

\usepackage{fullpage}

\begin{document}

\title{FlappBall Report}
\author{Benjamin Gunton, Harjuno Perwironegoro, Nimesh Subedi, Szymon Zmyslony}

\maketitle

\section{What and why?}
All of us have been always inspired and amazed by the enjoyment that a 2-button game with graphic from 80's can bring. At the same time, we wanted our extension to both teach us further concepts in C(extensive use of different libraries) and be interesting in implementing. Thus, we decided to create a two player game inspired by famous Flappy Birds. Each playeer controls one bird with two buttons (left, right) and tries to score a goal. The game is time limited and comes with sound effects, board that dispay the score, and menu to choose your character.
\section{Design}


\section{Testing}
The final game has been tested on the PI and worked quite smootly; however, major parts of testings have occured before it. We developed componenets of the project seperately and tested them so. Input buttons were tested using simple print message directly on the pi. Physics engine was tested numerically using assertions and print statements. All the behavoiur of objects inside the game was tested with the use of debugDraw funtion that drew hit boxes hidden behind images. Before transferring the game on the Pi, we succesfuly run in on Linux machine as it was easier to manipulate the code using IDE's and other more advanced tools. 
\section{Group reflection}
\subsection{How did we do?}
It our shared opinion that we have done very well as a group. Having read and discussed thoroughly  the spec during the first day of the project has proven to be very beneficial for us as it has developed deeper understanding of the problem (which was triggered by all of us asking questions and making other think). The formula in which half of the group debugs one part and the other half moves on to the next part was quite efficient and we do not plan to change in further projects. At the end, we have benefited the most from the group chat on Facebook which provided us with means to discuss our individual programming challenges on the go as well as coordinate our group tasks very dynamically. 
\subsection{How can we improve?}
The major area of improvement is splitting the work between group members. As some parts of assembly and emulator (Data processing) were much longer and harder than others, our initial plan had to be modified. It did not lead to a major downtime, but we still could have been more efficient had we thought about it at the beginning. Also, we feel that our extension could have been better, had we started to develop it earlier. Thus, in future, we plan on finishing the compulsory work earlier so we would get more time for improving our final project.
\section{Individual reflections}

\subsection{Ben}
\subsection{Uno}
\subsection{Nimesh}
\subsection{Szymon}
Based on my WEBPA feedback, I can tell that my group members have had quite positive experience in working with me. I happen to believe the same about them. I think we created a well-oiled team that was capable of helping each other with problems as well as stimulating and requiring hard work. This team expierence has definitely developed my skills of communicating with others and taught me how to efficiently divide work between team members. This team experience has definitely developed my skills of communicating with others and taught me how to efficiently divide work between team members.
\end{document}
